\label{fig:Beta_SWI_byS04_Stor} Posterior distributions of the
slope \(\beta_{G}\) in the four cardinal quadrants around the flux
tower, estimated without any \emph{a priori} knowledge of the surface
type classification or its spatial distribution. The NW, NE, SW \& SE
quadrants are denoted \(s_{1} \dots s_{4}\), so the panel layout
corresponds to the spatial configuration, with the tower at the centre.
\(\beta_{G}\) is higher in the all-land quadrants (\(s_{1}\) and
\(s_{3}\)), lowest in the all-water quadrant (\(s_{2}\)), and
intermediate in the mixed quadrant (\(s_{4}\)). The model thus correctly
retrieves the spatial pattern that we know to be present in the data.
The posterior distributions are narrow, meaning we have high certainty
about these parameter values.