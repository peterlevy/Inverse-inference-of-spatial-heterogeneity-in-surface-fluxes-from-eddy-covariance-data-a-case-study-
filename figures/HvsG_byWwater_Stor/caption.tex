\label{fig:HvsG_byWwater_Stor} Relationship between sensible
heat flux and incoming global irradiance, as influenced by the
proportion of surface water in the flux footprint \(W_{water}\) at the
Stordalen flux tower. On land, a relatively constant fraction of
incoming radiation is converted to sensible heat, producing a clear
relationship in the green data points, described by the slope
\(\beta_{G-\mathrm{land}}\). At night, the land cools via sensible heat
loss, so the intercept is negative. On water, most of the incoming
radiation is converted to latent heat (i.e.~evaporation), and there is
no strong dependence of H on G in the dark blue points, meaning the
slope \(\beta_{G-\mathrm{water}}\) is close to zero.