\section*{References}\label{references}
\addcontentsline{toc}{section}{References}

\hypertarget{refs}{}
\hypertarget{ref-Aubinet2002}{}
Aubinet M, Heinesch B, Longdoz B (2002) Estimation of the carbon
sequestration by a heterogeneous forest: Night flux corrections,
heterogeneity of the site and inter-annual variability. \emph{Global
Change Biology}, \textbf{8}, 1053--1071.

\hypertarget{ref-Aubinet2012}{}
Aubinet M, Vesala T, Papale D (eds.) (2012) \emph{\textup{Eddy
Covariance: A Practical Guide to Measurement and Data Analysis}}.
Springer Netherlands, pp.

\hypertarget{ref-Baldocchi2001}{}
Baldocchi D, Falge E, Gu L et al. (2001) FLUXNET: A new tool to study
the temporal and spatial variability of ecosystem-scale carbon dioxide,
water vapor, and energy flux densities. \emph{Bulletin of the American
Meteorological Society}, \textbf{82}, 2415--2434.

\hypertarget{ref-Bates2015}{}
Bates D, Mächler M, Bolker B, Walker S (2015) Fitting Linear
Mixed-Effects Models Using lme4. \emph{Journal of Statistical Software,
Articles}, \textbf{67}, 1--48.

\hypertarget{ref-Betancourt2017}{}
Betancourt M (2017) A Conceptual Introduction to Hamiltonian Monte
Carlo. \emph{arXiv:1701.02434 {[}stat{]}}.

\hypertarget{ref-Bolker2009}{}
Bolker BM, Brooks ME, Clark CJ, Geange SW, Poulsen JR, Stevens MHH,
White J-SS (2009) Generalized linear mixed models: A practical guide for
ecology and evolution. \emph{Trends in Ecology \& Evolution},
\textbf{24}, 127--135.

\hypertarget{ref-Davidson2016}{}
Davidson SJ, Sloan VL, Phoenix GK, Wagner R, Fisher JP, Oechel WC, Zona
D (2016) Vegetation Type Dominates the Spatial Variability in CH4
Emissions Across Multiple Arctic Tundra Landscapes. \emph{Ecosystems},
\textbf{19}, 1116--1132.

\hypertarget{ref-Finkelstein2001}{}
Finkelstein PL, Sims PF (2001) Sampling error in eddy correlation flux
measurements. \emph{Journal of Geophysical Research: Atmospheres},
\textbf{106}, 3503--3509.

\hypertarget{ref-Finnigan2003}{}
Finnigan JJ, Clement R, Malhi Y, Leuning R, Cleugh HA (2003) A
Re-Evaluation of Long-Term Flux Measurement Techniques Part I: Averaging
and Coordinate Rotation. \emph{Boundary-Layer Meteorology},
\textbf{107}, 1--48.

\hypertarget{ref-Gelman2013}{}
Gelman A, Carlin JB, Stern HS, Dunson DB, Vehtari A, Rubin DB (2013)
\emph{\textup{Bayesian Data Analysis, Third Edition}}, 3 edition edn.
Chapman and Hall/CRC, Boca Raton, pp.

\hypertarget{ref-Giljum2014}{}
Giljum M (2014) \emph{\textup{Object-Based Classification of Vegetation
at Stordalen Mire near Abisko by using High-Resolution Aerial Imagery}}.
MSc, Lund University, Lund, pp.

\hypertarget{ref-Horst1992}{}
Horst TW, Weil JC (1992) Footprint estimation for scalar flux
measurements in the atmospheric surface layer. \emph{Boundary-Layer
Meteorology}, \textbf{59}, 279--296.

\hypertarget{ref-IPCC2013}{}
IPCC (2013) \emph{\textup{Climate Change 2013: The Physical Science
Basis. Contribution of Working Group I to the Fifth Assessment Report of
the Intergovernmental Panel on Climate Change}}. Cambridge University
Press, Cambridge, United Kingdom; New York, NY, USA, pp.

\hypertarget{ref-Jammet2017}{}
Jammet M, Dengel S, Kettner E, Parmentier F-JW, Wik M, Crill P, Friborg
T (2017) Year-round CH4 and CO2 flux dynamics in two contrasting
freshwater ecosystems of the subarctic. \emph{Biogeosciences},
\textbf{14}, 5189--5216.

\hypertarget{ref-Johansson2006}{}
Johansson T, Malmer N, Crill PM, Friborg T, Åkerman JH, Mastepanov M,
Christensen TR (2006) Decadal vegetation changes in a northern peatland,
greenhouse gas fluxes and net radiative forcing. \emph{Global Change
Biology}, \textbf{12}, 2352--2369.

\hypertarget{ref-Jones2016}{}
Jones SK, Helfter C, Anderson M et al. (2016) The nitrogen, carbon and
greenhouse gas budget of a grazed, cut and fertilised temperate
grassland. \emph{Biogeosciences Discuss.}, \textbf{2016}, 1--55.

\hypertarget{ref-Kirschke2013}{}
Kirschke S, Bousquet P, Ciais P et al. (2013) Three decades of global
methane sources and sinks. \emph{Nature Geoscience}, \textbf{6},
813--823.

\hypertarget{ref-Kljun2015}{}
Kljun N, Calanca P, Rotach MW, Schmid HP (2015) A simple two-dimensional
parameterisation for Flux Footprint Prediction (FFP). \emph{Geosci.
Model Dev.}, \textbf{8}, 3695--3713.

\hypertarget{ref-Kormann2001}{}
Kormann R, Meixner FX (2001) An analytical footprint model for
non-neutral stratification. \emph{Boundary-Layer Meteorology},
\textbf{99}, 207--224.

\hypertarget{ref-Kumar2016}{}
Kumar J, Hoffman FM, Hargrove WW, Collier N (2016) Understanding the
representativeness of FLUXNET for upscaling carbon flux from eddy
covariance measurements. \emph{Earth System Science Data Discussions},
1--25.

\hypertarget{ref-Laird1982}{}
Laird NM, Ware JH (1982) Random-effects models for longitudinal data.
\emph{Biometrics}, \textbf{38}, 963--974.

\hypertarget{ref-Leclerc2014}{}
Leclerc MY, Foken T (2014) \emph{\textup{Footprints in Micrometeorology
and Ecology}}. Springer-Verlag, Berlin Heidelberg, pp.

\hypertarget{ref-Lee2006}{}
Lee X, Massman W, Law B (2006) \emph{\textup{Handbook of
Micrometeorology: A Guide for Surface Flux Measurement and Analysis}}.
Springer Science \& Business Media, pp.

\hypertarget{ref-Lehner2004}{}
Lehner B, Döll P (2004) Development and validation of a global database
of lakes, reservoirs and wetlands. \emph{Journal of Hydrology},
\textbf{296}, 1--22.

\hypertarget{ref-Levy2011}{}
Levy PE, Gray A, Leeson SR et al. (2011) Quantification of uncertainty
in trace gas fluxes measured by the static chamber method.
\emph{European Journal of Soil Science}, \textbf{62}, 811--821.

\hypertarget{ref-Levy2012}{}
Levy PE, Burden A, Cooper MDA et al. (2012) Methane emissions from
soils: Synthesis and analysis of a large UK data set. \emph{Global
Change Biology}, \textbf{18}, 1657--1669.

\hypertarget{ref-Levy2017}{}
Levy PE, Cowan N, van Oijen M, Famulari D, Drewer J, Skiba U (2017)
Estimation of cumulative fluxes of nitrous oxide: Uncertainty in
temporal upscaling and emission factors. \emph{European Journal of Soil
Science}, \textbf{68}, 400--411.

\hypertarget{ref-Malmer2005}{}
Malmer N, Johansson T, Olsrud M, Christensen TR (2005) Vegetation,
climatic changes and net carbon sequestration in a North-Scandinavian
subarctic mire over 30 years. \emph{Global Change Biology}, \textbf{11},
1895--1909.

\hypertarget{ref-Myhre2013}{}
Myhre G, Shindell D, Bréon F-M et al. (2013) Anthropogenic and Natural
Radiative Forcing. In: \emph{Climate Change 2013: The Physical Science
Basis. Contribution of Working Group I to the Fifth Assessment Report of
the Intergovernmental Panel on Climate Change} (eds Stocker T, Qin D,
Plattner G-K, Tignor M, Allen S, Boschung J, Nauels A, Xia Y, Bex V,
Midgley P), pp. 659--740. Cambridge University Press, Cambridge, United
Kingdom; New York, NY, USA.

\hypertarget{ref-Papale2006}{}
Papale D, Reichstein M, Aubinet M et al. (2006) Towards a standardized
processing of Net Ecosystem Exchange measured with eddy covariance
technique: Algorithms and uncertainty estimation. \emph{Biogeosciences},
\textbf{3}, 571--583.

\hypertarget{ref-Peek2002}{}
Peek M, Russek-Cohen E, Wait A, Forseth I (2002) Physiological response
curve analysis using nonlinear mixed models. \emph{Oecologia},
\textbf{132}, 175--180.

\hypertarget{ref-Pinheiro2006}{}
Pinheiro J, Bates D (2006) \emph{\textup{Mixed-effects models in S and
S-PLUS}}. Springer Science \& Business Media, pp.

\hypertarget{ref-Schaepman-Strub2009}{}
Schaepman-Strub G, Limpens J, Menken M, Bartholomeus HM, Schaepman ME
(2009) Towards spatial assessment of carbon sequestration in peatlands:
Spectroscopy based estimation of fractional cover of three plant
functional types. \emph{Biogeosciences}, \textbf{6}, 275--284.

\hypertarget{ref-Schmid1990}{}
Schmid HP, Oke T (1990) A model to estimate the source area contributing
to turbulent exchange in the surface layer over patchy terrain.
\emph{Quarterly Journal of the Royal Meteorological Society},
\textbf{116}, 965--988.

\hypertarget{ref-Schuepp1990}{}
Schuepp PH, Leclerc MY, MacPherson JI, Desjardins RL (1990) Footprint
prediction of scalar fluxes from analytical solutions of the diffusion
equation. \emph{Boundary-Layer Meteorology}, \textbf{50}, 355--373.

\hypertarget{ref-StanDevelopmentTeam2016}{}
Stan Development Team (2016) Rstanarm: Bayesian applied regression
modeling via Stan.

\hypertarget{ref-Sturtevant2012}{}
Sturtevant CS, Oechel WC, Zona D, Kim Y, Emerson CE (2012) Soil moisture
control over autumn season methane flux, Arctic Coastal Plain of Alaska.
\emph{Biogeosciences}, \textbf{9}, 1423--1440.

\hypertarget{ref-vanOijen2017}{}
van Oijen M, Cameron D, Levy, Peter E., Preston, Rory (2017) Correcting
Errors from Spatial Upscaling of Nonlinear Greenhouse Gas Flux Models.
\emph{Environmental Modelling and Software}, \textbf{94}, 157--165.

\hypertarget{ref-Xu2006}{}
Xu L, Furtaw MD, Madsen RA, Garcia RL, Anderson DJ, McDermitt DK (2006)
On maintaining pressure equilibrium between a soil CO2 flux chamber and
the ambient air. \emph{Journal of Geophysical Research}, \textbf{111}.

\hypertarget{ref-Zona2009}{}
Zona D, Oechel WC, Kochendorfer J et al. (2009) Methane fluxes during
the initiation of a large-scale water table manipulation experiment in
the Alaskan Arctic tundra: WATER TABLE IMPACTS ON METHANE FLUXES.
\emph{Global Biogeochemical Cycles}, \textbf{23}, n/a--n/a.