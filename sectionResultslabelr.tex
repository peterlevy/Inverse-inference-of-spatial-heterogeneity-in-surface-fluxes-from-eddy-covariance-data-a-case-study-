\section{Results}\label{results}

Figure \ref{fig:HvsG_byWwater_Stor} shows the basis of our analysis of
sensible heat fluxes to demonstrate the method. On land, a relatively
constant fraction of incoming radiation is converted to sensible heat,
producing a clear relationship in the green data points, described by
the slope \(b_G\). At night, the land cools via sensible heat loss, so
the intercept is negative. On water, most of the incoming radiation is
converted to latent heat (i.e.~evaporation), and there is no strong
dependence of \(H\) on \(G\) in the dark blue points, meaning the slope
\(b_G\) is close to zero over water. Without using any \emph{a priori} 
knowledge of the surface type classification or its spatial
distribution, we can use to Equation \ref{eq:pred_Fbarhat} to estimate
the slope \(b_G\) in the four cardinal quadrants around the flux tower.
In this simplest case, we set: \(n_u = 1\) (i.e.~no distinction between
surface types) so the only fixed effect is a single intercept \(\beta\),
\(n_s = 4\) (i.e.~the four cardinal quadrants), and use a single
temporal covariate \(\mathbf{Z} = G\). With this, we fit the random
slope term \(b_G\) for each quadrant, and with the Bayesian methodology,
we estimate their posterior distributions (Figure
\ref{fig:Beta_SWI_byS04_Stor}). This shows that \(b_G\) is higher in the
all-land quadrants (\(s_{1}\) and \(s_{3}\)), lowest in the all-water
quadrant (\(s_{2}\)), and intermediate in the mixed quadrant
(\(s_{4}\)). The model thus correctly retrieves the spatial pattern that
we know to be present in the data. The posterior distributions are
narrow, meaning we have high certainty about these parameter values.

We can repeat this with a further subdivision into 16 subquadrants (grey
lines in Figure 1)(Figure \ref{fig:Beta_SWI_byS16_Stor}). Here, the
posterior distributions of the slope \(b_G\) are still narrowly-defined
in the regions where the footprint gives high weightings, and generally
yields the known correct result (high values on land, low values on
water). Because the predominant wind directions are easterly and
westerly, footprint weightings are low in the subquadrants to the north
and south (top and bottom rows). Consequently, parameter estimates for
these less frequently sampled regions are uncertain, the distributions
are wide, and even the sign of the parameter values is uncertain. There
is an apparent problem of mirroring, where adjacent subquadrants have
positive and negative values, which sum to a plausible mean. This can be
removed by stipulating a stronger prior which makes these less
plausible, adding a spatial autocorrelation term (stating that similar
values will tend to occur together), or more simply, using the spatial
surface type distribution.

Applying the model in this last way, we can perform a more sophisticated
analysis which uses the spatial distribution of surface types, and uses
information-theoretic criteria to discern the appropriate number of
spatial random effect terms to include. In this case, we set:
\(n_u = 2\) (i.e.~using the map of water and land), a single temporal
covariate \(\mathbf{X} = G\), and fixed effects for the slopes
\(\beta_{G-water}\) and \(\beta_{G-land}\). To this we added either a
single intercept, or random intercepts \(b_s\) for \(n_s = 4\) or
\(n_s = 16\) as above.

The posterior distributions of the slope parameters \(\beta_{G}\) are
shown in Figure \ref{fig:Beta_SWI_byS0S04S16_Stor}. This yields
plausible values, with a positive slope on land and a near-zero slope
over water, as we know to be correct from the data in Figure
\ref{fig:HvsG_byWwater_Stor}. The distributions show the values are
better constrained for \(\beta_{G-land}\) (within a range of
\(\approx 0.02\)) than for \(\beta_{G-water}\) (range of
\(\approx 0.04\)), as expected from the greater scatter in the data over
water in Figure \ref{fig:HvsG_byWwater_Stor}. The values vary depending
on the number of spatial random effects \(n_s\) which are included
(i.e.~granularity of the spatial heterogeneity that is recognised). We
can assess how many it is justified to include on the basis of the
information criteria in Table \ref{tab:bic_output}. This shows that,
when the surface type classification is used (i.e. \(n_u > 1\)), BIC is
larger when \(n_s = 16\) compared to when \(n_s = 4\), so using the
higher number of spatial regions is not justified. This bears out what
we saw in Figure x, where parameter estimates in the furthermost
subquadrants were unreliable. The minimum value of BIC is obtained with
\(n_u = 5\) and \(n_s = 4\), suggesting that the distinctions between
surface types in the more detailed classification (not just water versus
land) are relevant to sensible heat fluxes. BIC values for the cases
\(n_u = 1, n_s = 16\), \(n_u = 5, n_s = 1\) and \(n_u = 2, n_s = 4\)
ranked next and were all similar.

We repeat a similar analysis for methane fluxes, where the underlying
model is much less well-known, although at least a reasonably clear
temperature response provides a similar basis (Figures
\ref{fig:Fch4vsT_byVeg_ch} and \ref{fig:Fch4vsT_byWwater_Stor}). Figure
\ref{fig:Fch4vsT_byVeg_ch} shows that there are at least some distinct
differences in methane fluxes between the different surface types. The
semi-wet class has a much lower mean than the graminoid and hummock
classes, but these latter are rather similar. The data are rather
variable, and a substantial part of this is point-to-point variability,
characterised by location but not to any measured covariate. There were
no chamber measurements in the wet or water surface types.

When the footprint was on the land, ecosystem-scale fluxes were similar
in magnitude to the mean of fluxes measured in the chambers at a
corresponding temperature (of the order of 100 nmol m\(^{-2}\)
s\(^{-1}\) at 10 \(^o\)C, Figure \ref{fig:Fch4vsT_byWwater_Stor}).
Methane fluxes increase with biological production in the peat,
producing a clear relationship in the green data points, described by
the slope \(\beta_{T_{s}-\mathrm{land}}\). The lake temperature is not
strongly coupled to soil temperature, and lake fluxes are also
influenced by physical turbulence (Jammet \emph{et al.} , 2017), so there
is no strong dependence of \(F_{CH_4}\) on \(T_s\) in the dark blue
points, meaning the slope \(\beta_{T_{w}-\mathrm{water}}\) is close to
zero.

As for sensible heat, we fit a simple model which includes the most
important temporal covariates, and assess whether we can discern the
expected spatial pattern in the residuals. To do this, we set:
\(n_u = 1\) (i.e.~no distinction between surface types), four temporal
covariates \(\mathbf{X} = {Ts, Tw, G, v}\), with fixed effect parameters
for the slopes \(\beta_{T_s}\), \(\beta_{T_w}\), \(\beta_{G}\), and
\(\beta_{v}\). To this we added either a single intercept, or random
intercepts \(b_s\) for \(n_s = 4\) or \(n_s = 16\) as above.

Figure \ref{fig:b_Fch4_U1S04_byS_Stor} shows that after removing the
temporal variation, the residual flux, \(b_{s}\), is higher in the
all-land quadrants (\(s_{1}, s_{3}\)), and lower in the water quadrants
(\(s_{2}, s_{4}\)), as expected based on the pattern in Figure
\ref{fig:Fch4vsT_byWwater_Stor}. The exception is that the mixed
water/land quadrat \(s=4\) has a lower value than the all-water quadrant
\(s=2\), though the two distributions largely overlap.

When we use 16 spatial regions in the same procedure, we find that
parameter estimates remain plausible and well-defined in regions near
the tower (Figure \ref{fig:b_Fch4_U1S16_byS_Stor}), but are highly
uncertain and implausible in some other cases (e.g. \(s=2,13,14\)), as
was the case for sensible heat.

Again, we can use information on the spatial distribution of surface
types in a more sophisticated analysis to explore the the appropriate
number of spatial random effect terms to include. Figure
\ref{fig:Beta_Fch4_byS0S04S16_Stor} shows the effect of including
\(n_s\) = 1, 4, or 16 random intercepts \(b_s\), when using the same
\(\mathbf{X}\) variables as describe above, but with \(n_u = 2\). In
this way, we fit separate \(\beta\) slope parameters for the effects of
\(T_w\) and \(v\) on fluxes from water, and the effects of \(T_s\) and
\(G\) on fluxes from land. Figure \ref{fig:Beta_Fch4_byS0S04S16_Stor}
shows the effect of including \(n_s\) = 1, 4, or 16 random intercepts
\(b_s\) on the estimates of the fixed-effect parameters. This indicates
that including the spatial random effects can have a sizeable impact on
parameter values. Including the random effects can make the values
higher or lower, but the effect is not straightforwardly predictable -
e.g.~with \(n_s\) = 4 and =16, the distributions for
\(\beta_{T_{w-\mathrm{water}}}\) are almost identical, whereas the
distributions for \(\beta_{G_{\mathrm{land}}}\) are on either side of
the distribution for \(n_s = 1\).

The information-theoretic criteria give an indication of how many terms
should be included in the model. As for sensible heat, the BIC minimum
indicates that \(n_u = 5, n_s = 4\) is the best configuration, with the
other \(n_u = 5\) combinations ranking second and third. The use of five
surface types to describe the variability in methane flux is thus
strongly supported. The related AIC and DIC criteria always produced the
same rankings (data not shown), but the alternative WAIC in fact
suggested \(n_u = 5, n_s = 16\) as the best configuration, so the
results are not completely clear-cut.

We investigated the influence of the prior by analysing two variations:
using the model fitted to the chamber data as the prior, or using
``weakly-informative'' priors with mean zero and wide standard
deviations (Stan Development Team, 2016). For the wet and water surface
types, there was no chamber data, so default weak priors were used here.
Distributions of the \(\beta\) temperature coefficient for the five
surface types are shown in Figure
\ref{fig:Beta_T_Fch4_U5S01_withPrior_Stor}.

Generally, the red and green curves are not very different, so the prior
does not have a very strong effect on the results. The posterior moves
substantially away from the prior in the case of the hummock, and
graminoid surface types. The priors for these are strongly influenced by
a small number of high fluxes in the chamber data. These are plausible
for individual point flux measurements, which often show a lognormal
distribution, but not plausible for the ecosystem-scale mean which eddy
covariance measures. The difference is thus partly due to the difference
in measurement scales, and could be accounted for explicitly in future
work (e.g.~see Levy \emph{et al.} , 2017).

The uncertainty in a given parameter estimate is characterised by the
spread of its posterior distribution, and this can be summarised as a
credibility interval or standard deviation, \(\sigma\), or its
reciprocal the precision (\(1 / \sigma\)). To examine the level of
uncertainty in our inference of spatial heterogeneity in residual
methane flux, Figure \ref{fig:precision_map} shows the plot of the
precision of our estimates of the spatial random intercept term, \(b_s\)
(from the model fitted as decribed above, with \(n_u = 2\) and
\(n_s = 16\)). This gives a quantitative representation of our intuitive
expectation, that we have higher certainty in flux estimates from
regions near the flux tower compared with those at a distance. It also
quantifies the effect of prevailing wind patterns on the extent to which
different regions are sampled by the footprint. In addition, this
incorporates the differing degrees of variability in fluxes from
different regions, and the extent to which these are explained by
temporal or spatial covariates.