\section{Methods}\label{methods}

\subsection{Site Description}\label{site-description}

Stordalen mire is a subarctic peatland located near Abisko, Northern
Sweden (68\(^{\circ}\) 20' N, 19\(^{\circ}\) 03' E). At a broad level,
the landscape is heterogeneouis, comprising freshwater lakes, smaller
pools, and non-inundated bog with a water table below the surface
(Figure \ref{fig:fp_on_map_Stordalen} and \ref{fig:r_water_land}). At a
finer level, the terrestrial vegetation types can be classified into:
(i) moss hummocks containing shrubs, (ii) semiwet (ombro-minerotrophic)
bog with a water table below the surface, (iii) wet (ombrotrophic) bog
pools with a water table at or near the surface, and (iv)
graminoid-dominated bog with a water table above the peat surface
(Malmer \emph{et al.} , 2005, Johansson \emph{et al.} (2006)). The
distribution of these vegetation types has been mapped using remote
sensing (Giljum, 2014)(Figure \ref{fig:r_u5}).

\subsection{Chamber flux measurements}\label{chamber-flux-measurements}

Static chamber measurements of CH\(_4\) flux were made in June, August
and September 2013. 46 collars were located across the mire, clustered
in five locations accessible from boardwalk. Circular PVC collars (40 cm
diameter), fitted with a flange, were inserted to a depth of 5 cm into
the ground prior to the start of measurements and remained in the ground
for the duration of the study. Cylindrical PVC chambers (40 cm diameter,
20 cm height) with a matching flange, were sealed on the collars for the
duration of each flux measurement (45 minutes). Each chamber lid had a
vent (pressure compensation plug; Mouser Electronics, London, UK) to
compensate for pressure changes whilst minimising diffusion losses (Xu
\emph{et al.} , 2006). During each flux measurement, gas samples were
extracted from the chamber headspace at 5, 15, 30 and 45 minutes via a
polypropylene syringe and a three-way stopcock. Samples were transferred
to 20-ml vials fitted with chlorobutyl rubber septa. The samples and
three sets of four certified standard concentrations were analysed on a
gas chromatograph (HP5890 Series II, Hewlett Packard, Agilent
Technologies, Stockport, UK) with a flame ionization detector. The limit
of detection was 0.07 ppm CH\(_4\). Peak integration was carried out
with Clarity chromatography software (DataApex, Prague, Czech Republic).
The flux was calculated from the change in mixing ratio within the
chamber headspace against time:

\begin{equation}
 F = \frac{\mathrm{d}C}{\mathrm{d}t_0} \cdot \frac{\rho V}{A}

\end{equation}

where \(F\) is the gas flux in \(\mathrm{nmol\,m^{-2}\,s^{-1}}\),
\(\mathrm{d}C/\mathrm{d}t_0\) is the initial rate of change in mixing
ratio with time in \(\mathrm{nmol\,mol^{-1}\,s^{-1}}\), \(\rho\) is the
density of air in \(\mathrm{mol\,m^{-3}}\), \(V\) is the volume of the
chamber in \(\mathrm{m^3}\) and \(A\) is the ground area enclosed by the
chamber in \(\mathrm{m^2}\). \(\mathrm{d}C/\mathrm{d}t_0\) was estimated
by linear and nonlinear regression as described in Levy et al. (2011).

\subsection{Eddy covariance flux
measurements}\label{eddy-covariance-flux-measurements}

Fluxes of CH\(_4\), CO\(_2\), H\(_2\)O and sensible heat were monitored
nearly continuously with an eddy covariance system located at the shore
of Villasjon (Fig. 1), as described in full by (Jammet \emph{et al.} ,
2017). The system comprised a 3-D sonic anemometer (R3-50, Gill
Instruments Ltd.), an open-path infrared gas analyser (IRGA) for
CO\(_2\) and H\(_2\)O (LI7500, LICOR Environment, NE, USA) and a
closed-path analyser for CH\(_4\) (FGGA, Los Gatos Research, CA, USA),
logged on a data logger (CR1000, Campbell Scientific, Inc., UT, USA).
The sonic and IRGA were mounted on a mast at 2.5 m, with a sample inlet
line to the FGGA adjacent to these. Fluxes were calculated using the
EddyPro version 5.2 open-source software (hosted by LICOR Environment,
USA). Processing of the 10-Hz raw data followed standard eddy covariance
procedures (Aubinet \emph{et al.} , 2012, Lee \emph{et al.} (2006)).
Fluxes were rejected for some time periods on the basis of quality
checks, including the number of spikes, skewness and kurtosis,
stationarity, turbulence (low u* values) and implausible time lags
(Papale \emph{et al.} , 2006). Although Jammet et al. (2017) imputed
missing values, only observed fluxes were analysed here. For every time
period with a valid flux measurement, we calculated the flux footprint
\(\mathbf{W}\) over a 200 x 200 m domain around the tower at 2-m
resolution. For simplicity, we use the commonly-used simple analytical
model of Kormann and Meixner (2001), but acknowledge that more
comprehensive models are available (Kljun \emph{et al.} , 2015).

\subsection{Statistical analysis}\label{statistical-analysis}

We consider two examples to demonstrate the application of the approach
to the Stordalen flux data. Firstly, we analyse the response of sensible
heat to incoming solar radiation. In this case, we have a clearly
defined spatial pattern, which we know in advance, because land and
water surfaces behave quite differently for clear physical reasons. We
consider only two surface type in this case (\(\mathbf{u}\) = \{water,
land\}). We use this to test the approach to detecting spatial
heterogeneity, where we have a good knowledge of the true pattern
\emph{a priori} . Secondly, we analyse the more complex case of methane
flux, where we define five different surface types (\(\mathbf{u}\) =
\{hummock, semi-wet, wet, graminoid, water\}), all with potentially
different responses of CH\(_4\) flux to multiple temporal covariates. In
both cases, we can sub-divide the domain, with increasing granularity,
into 4, 16 or 64 spatial regions (the cardinal quadrants centred on the
tower, and further repeated subdivisions of these).

Applying Equations \ref{eq:sWtMean} and \ref{eq:uWtMean} provides the
footprint weightings for each defined spatial region \(\mathbf{W}_{s}\)
and each surface type \(\mathbf{W}_{u}\). Equation \ref{eq:pred_Fbarhat}
gives the prediction for what we expect to observe by eddy covariance at
time \(t\), given a calculated footprint \(\mathbf{W}\), a known
distribution of surface types \(\mathbf{U}\), a pre-defined matrix of
spatial regions \(\mathbf{S}\), the autoregressive term \(\phi\), and
the parameter vectors for fixed and random effects,
\(\boldsymbol{\beta}\) and \(\mathbf{b}\). The likelihood of a series of
\(n_t\) observations of \(\widehat{\bar{F_t}}\) arising from normal
distributions with means determined by Equation \ref{eq:pred_Fbarhat}
is:

\begin{equation}
 \mathcal{L} = \prod_{t=1}^{n_t} \frac{1}{\sigma_{t}^\mathrm{obs} \sqrt{2\pi}} \mathrm{exp}(-(\widehat{\bar{F_t}}^\mathrm{obs} - \widehat{\bar{F_t}}^\mathrm{pred})^2/2 \sigma_{t}^{\mathrm{obs}^2})

\end{equation}

where \(\sigma_{t}^{\mathrm{obs}}\) is the uncertainty in the observed
(eddy covariance) flux at each time point. The posterior distribution
for \(\beta\), \(b\) and the predicted flux were estimated using
Hamiltonian MCMC, using the algorithm implemented in the R package
\(\texttt{rstanarm}\) (Stan Development Team, 2016, Betancourt (2017)).
Initial fits were obtained by the restricted maximum likelihood method
using the \(\texttt{lme4}\) package in R (Bates \emph{et al.} , 2015).

We based the choice of independent variables to use in modelling the
CH\(_4\) flux on the earlier analysis of (Jammet \emph{et al.} , 2017).
Although many of the variables are strongly correlated, this suggested
that CH\(_4\) fluxes from the lake surface were controlled by water
temperature (as a control on biological production) and turbulence (as a
physical mechanism driving the flux at the water surface). CH\(_4\)
fluxes from the land surface were controlled by soil temperature, with
additional effects of solar radiation and humidity (as drivers of
stomatal opening and transport through plant pathways). The chamber flux
data were used to establish the prior distribution for model parameters.
The chambers were allocated to the same set of surface types, and the
same model was fitted to the chamber flux data for each of these groups,
in the same way as for the eddy covariance data. The posterior
distribtion of parameters established in this way was then used as the
prior distribution in the analysis of the eddy covariance data. The
uncertainty in each half-hourly flux observation was estimated using the
method of Finkelstein and Sims (2001). The effect of including different
terms in the model fitted, and the granularity of the grid on which
spatial heterogeneity was resolved, was assessed using common model
selection criteria: AIC, BIC, and DIC. To allow upscaling of
measurements to a wider region, the model fitted to observations of
\(\widehat{\bar{F_t}}\) was used to make explicit predictions of
\(\bar{F_t}\) over a larger spatial domain, where the surface type
classification was available (Giljum, 2014).